
\section{Motivación}
\label{intro:motivacion}
%\del{\todo{o pones citas o no haces estas aseveraciones} El uso de simuladores de realidad virtual en medicina es muy beneficioso al poder crear entornos seguros y reproducibles para los usuarios\todo{ una cita}. El auge que han tenido estos simuladores es gracias a su idoneidad para el entrenamiento y la enseñanza de los procedimientos médicos, consiguiendo una gran aceptación en todos los campos de la medicina\todo{cita}.  Aun así, la diversidad de procedimientos y la complejidad del cuerpo humano y llevar a cabo cada tipo de simulación diferente son las principales dificultades que se presentan en un futuro a corto plazo.  Los principales problemas que se han encontrado en los diferentes simuladores médicos son los siguientes:}

%\todo{ 1. tienes que hablar de la importancia de tener una libreria de pacientes 2. de la necesidad adapatarlos a la pose de operación y 3 de las aplicacioens que necesitan modificar la pose en tiempo real.  Habla de los problemas de las ténicas de posing existentes. tienes que hablar de los problemas de posing a la hora crear librerías validas para un determinado procedimiento con una poses estática o la importancia de poder cambiar la pose en el procedimiento como con los x-ray. Indicar los problemas a la hora de modificar geometrias incompletas sin descripcion de comportamiento....}
%\todo{He visto que mucho de lo que te he dicho está en los párrafos siguientes. Reestructúralo y lo vuelvo a corregir. Recuerda. Todo lo que quieres que me vuelva a leer ponlo con new.}


En el contexto de la simulación para entrenamiento médico, es importante que un profesional sanitario en formación se enfrente a la mayor cantidad de casos y  variaciones anatómicas posibles. En general, los simuladores médicos utilizan modelos anatómicos propios y específicos que no representan toda la variabilidad posible para entrenar de forma adecuada un determinado procedimiento. 
%\new{y es por ello que su utilización quedaría sesgada y resultaría complicado generalizar este aprendizaje al ámbito médico real }.
Por ello, nuevas plataformas de entrenamiento médico están incorporando datos de pacientes reales en sus simuladores \cite{Willaert2012,Votta2013}. Este enfoque no está exento de problemas:
\begin{enumerate}
   %\item Los datos anatómicos de los pacientes virtuales son estáticos y representan poca variabilidad.
    %\item En cada simulador se suele crear modelos anatómicos específicos y no son transferibles entre ellos.
    \item \new{Ningún método de adquisición de imagen médica es capaz de registrar todos los tejidos de un paciente.\todo{yo pondría ejemplos} Si bien es cierto que se podrían combinar datos procedentes de distintas técnicas es poco habitual dado que se presentan problemas de: coste, registro de imágenes, técnicas contraindicadas para algunos pacientes (por ejemplo el \ac{TC} abdominal en embarazadas)...}
    \item A partir las distintas técnicas de imagen médica es difícil obtener las propiedades mecánicas de todos o alguno de los tejidos registrados.
    \item Es habitual que las imágenes médicas se capturen con en paciente en una posición distinta a la requerida en la intervención.
    \item No siempre están disponibles información específica del paciente o solo se trata de una zona concreta del paciente. \todo{No entiendo este punto}
\end{enumerate}
%
Por todo lo anterior, el objetivo de \ac{RASimAs},  no es crear una base de datos de pacientes reales, sino de pacientes virtuales en la que estén representadas el mayor numero de representaciones anatómicas posibles. Estos pacientes virtuales se construirán promediando datos provenientes de pacientes reales. Puesto que no se trata de ensayar el procedimiento en un paciente real, el método, que adapte la pose de los pacientes virtuales a la requerida en el procedimiento, no tiene porque ser precisó desde el punto de vista fisco. Basta con que el resultado sea plausible, permitiendo al médico en formación entrenar de forma efectiva.
%\del{Por tanto, se hace necesario disponer de algún método que pueda capturar variaciones anatómicas, de tal manera que pueda incorporarlas al simulador que se estuviera utilizando. En este caso, en lugar de tener un modelo anatómico con una configuración concreta, % En la motivación, cuando hables de la base de datos, di que en muchas aplicaciones lo importante es la variabilidad anatómica, no trabajar con pacientes reales. }
%basta con poder transformar cualquier modelo de paciente virtual existente a una pose plausible, pudiéndose utilizar en cualquier simulador que así lo requiera. De esta forma, se intenta abarcar la mayor cantidad posible de situaciones que puedan ayudar a los aprendices a mejorar su entrenamiento y por tanto su profesionalidad .}
 
Cabe destacar que, dentro de la generación de imágenes por computador, existen diferentes técnicas para poder animar personajes virtuales. 
\del{Es habitual que estos métodos sean utilizados para animar estructuras anatómicas con diversas finalidades como pueden ser: cinematográfica, lúdica o sanitarias.}\todo{esta frase no es verdad.} 
Estas técnicas suelen estar divididas comúnmente entre métodos geométricos y métodos basados en física. Ambos enfoques presentan ventajas \new{e inconvenientes} \del{sobre los otros} y su uso \new {dependerá de} \del{se extiende en base a} sus características. 

Considerando los problemas citados anteriormente y \new{las restricciones impuestas por el proyecto \ac{RASimAs} (ver apartado \ref{XXXXXXXX}), esta tesis pretende comprobar si es posible diseñar un método geométrico que permita entrenar en el entrono medico de forma efectiva.}\del{dentro del contexto del proyecto europeo \ac{RASimAs}, la motivación de esta tesis es comprobar si es posible diseñar un método geométrico que pueda solventar determinados problemas.} A continuación, se muestran las diferencias \new{entre ambas aproximaciones}\del{que tienen las técnicas geométricas frente a los métodos basados en físicas:}

\begin{itemize}
\item Los métodos basados en \new{modelos físicos}\del{físicas} se centran en conseguir deformaciones \new{plausibles}\del{realistas REALISTAS NO QUIERE DECIR NADA}, frente a los geométricos que proporcionan soluciones plausibles. \new{Es decir, deformaciones que el usuario pueda interpretar como reales.} 
%\todo{esto es una ventaja? quizás habría que poner arriba en vez de ventajas, diferencias}
%OK


\item Los algoritmos geométricos son más robustos\new{. En la mayoría de los casos son soluciones \emph{ad hoc} que no sufren de los problemas de estabilidad de los métodos numéricos que se utilizan para resolver las ecuaciones diferenciales que modelan el comportamiento de los distintos objetos físicos. }\del{ante la falta de una descripción completa de un modelo anatómico. Debido a las dificultades que tienen la adquisición de imágenes médicas, no es habitual que todas las estructuras anatómicas estén presentes o se puedan capturar el comportamiento de todas ellas.}

\iten \new{Por otro lado, los métodos basados física requieren caracterizar mecánicamente los modelos que se van a simular. Está información no siempre puede obtenerse de forma precisa a partir de las imágenes médicas.}

\item \new{Debido a la complejidad de los comportamientos que deben simularse, la mayoría de los algoritmos biomecánicos actuales se centran en el comportamiento de estructuras concretas, como por ejemplo los sistemas de animación musculo-esqueleto que obvian las interacciones con otras estructuras. }

\item Es habitual que los métodos geométricos tengan tasas de refresco altas, en comparación a los algoritmos basados en física\del{s}. Estas técnicas habitualmente son demasiado costosas para conseguir interactividad\del{, sea necesario la intervención de un usuario dado su complejidad o se centren en una zona limitada de la anatomía??????????????????????}.\new{El problema se agrava  cuanto más complejos sean los comportamientos que se quieran simular. Destacar que la simulación biomédica tiene un alto grado de complejidad debido a la variedad de estructuras y comportamientos mecánicos a simular.}

\item \del{Debido a su popularidad, actualmente los métodos geométricos están diseñados para aprovechar totalmente la arquitectura de las tarjetas gráficas.}
    
\item \del{Los modelos basados en físicas están muy extendidos, aunque en su gran mayoría se centran solo en el sistema \new{músculo-esquelético} \del{musculoesqueletal}.} \todo{El sistema locomotor, llamado también sistema músculo-esquelético.}
\end{itemize}

Disponer de un método geométrico que permita adaptar la postura a modelos anatómicos humanos ayudará a:

\begin{itemize}
    \item Poder modificar cualquier modelo virtual con estructuras anatómicas internas a la posición requerida por un procedimiento médico.
    \item \new{Poder adaptar cualquier pacientes virtual aunque este presenten tejidos incompletos o cuyas propiedades mecánicas no hayan sido registradas.}
    \item \del{El método permite una interacción inmediata y servirá para utilizarse en simuladores de entrenamiento médico????}
    \item \todo{Su flexibilidad y su rapidez le hacen posible incorporarlo en cualquier simulador o herramienta de \ac{RV}. REDACTA BIEN}
\end{itemize}


\section{Hipótesis de partida} 
\label{intro:hipotesis}
%\todo{la hipótesis de partida es buena}
Con las motivaciones anteriormente descritas, se procede a formular la siguiente hipótesis de partida:

%\emph{es posible crear un método geométrico para animar modelos anatómicos humanos en tiempo real del que no se posean completamente todos sus tejidos o propiedades mecánicas que pueda ser utilizado en simuladores médicos para entrenar nuevos médicos. }

%\emph{ En el contexto del entrenamiento en simuladores médicos, se pueden diseñar algoritmos geométricos capaces de adaptar la antinomia del paciente virtual a la pose requerida en el procedimiento simulado de forma que la transferencia de competencias sea efectivo.}

\begin{center}
    \begin{minipage}{0.9\linewidth}
        %\vspace{5pt}%margen superior de minipage
        {\small
\emph{es posible diseñar un algoritmo geométrico capaz de adaptar modelos anatómicos de un paciente virtual desde una posición inicial a la pose requerida por un procedimiento médico y que sirvan para que la transferencia de competencias sea efectiva en el contexto de formación utilizando simuladores de realidad virtual. }
        }
        %\vspace{5pt}%margen inferior de la minipage
    \end{minipage}
    
    
\end{center}




% Hipótesis:
% -        Método geométrico puede ser utilizado en el contexto de los médicos:
% o   Datos incompletos (geométricos y mecánicos)
% o   Velocidad para el tiempo real. 
%o   Otras ideas